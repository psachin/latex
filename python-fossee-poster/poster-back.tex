\nonstopmode
%==============================================================================
%== template for LATEX poster =================================================
%==============================================================================
%
%--A0 beamer slide-------------------------------------------------------------
\documentclass[final]{beamer}

\usepackage{graphicx}
\usepackage{xcolor}
\usepackage[orientation=portrait,size=a0,
            scale=2.2         % font scale factor
           ]{beamerposter}
           
\geometry{
  hmargin=2.5cm, % little modification of margins
}

%
\usepackage[utf8]{inputenc}
\usepackage{ragged2e}

\linespread{1.15}
%
%==The poster style============================================================
\usetheme{sharelatexback}

%==Title, date and authors of the poster=======================================
%\title
%[http://python.fossee.in, info@fossee.in, python@fossee%.in] % Conference
%{ % Poster title
%}


%\author{ % Authors
%Author One\inst{1}, Author Two\inst{2}, Author Three\inst{2,3}
%}
%\institute
%[Very Large University] % General University
%{
%\inst{1} Very Large University, Neverland
%\inst{2} Other University, Neverland
%\inst{3} Yet Another University, Neverland
%}
%\date{\today}



\begin{document}
\begin{frame}[t]
%==============================================================================
\begin{multicols}{2}
%==============================================================================
%==The poster content==========================================================
%==============================================================================

%% BACK

\section{How can you learn Python ?}
\begin{itemize}
\item \justifying {{\bf{Spoken Tutorial}} - The FOSSEE project has created a series of Spoken Tutorials on
Python.  These are available for learning, on the Spoken Tutorial
website, free of cost. You can access these tutorials from this
link: \\ {\color{blue}{\fontfamily{lmss}\selectfont python.fossee.in/spoken-tutorials}}} \par

\vskip5ex

\begin{figure}
\centering
\fboxsep=0mm%padding thickness
\fboxrule=1pt%border thickness
\fcolorbox{gray}{white}{\includegraphics[width=1.0\columnwidth]{st.png}}
\texttt{\small Spoken Tutorial website}
%\caption{Spoken tutorial website}
\end{figure}

\vskip5ex

\item \justifying{{\bf{Textbook Companion Internship}} - Learn Python
  in a practical way by contributing to the Python Textbook Companion
  Internship. It aims to create Companions by coding solved examples
  from Standard textbooks, using Python. Participate and earn
  attractive honorarium and Certificate of Internship from FOSSEE, IIT
  Bombay!  For more details, please visit: \\
  {\color{blue}{\fontfamily{lmss}\selectfont
  python.fossee.in/textbook-companion-project}}}\par

\vskip5ex

\begin{figure}
\centering
\fboxsep=0mm%padding thickness
\fboxrule=1pt%border thickness
\fcolorbox{gray}{white}{\includegraphics[width=0.9\columnwidth]{tbc.png}}
\texttt{\small Python Textbook Companion website}
%\caption{Python textbook companion}
\end{figure}


\vskip5ex

\item {{\bf{SELF Workshops}} - The Spoken Tutorial Team conducts
  workshops on Python.  These are completely free of cost, and are
  conducted without the need of any domain expert. Learn Python and
  obtain a certificate from Spoken Tutorial Project, IIT Bombay, upon
  successful completion of the post-workshop evaluation test. Please
  visit: {\color{blue}{\fontfamily{lmss}\selectfont
  python.fossee.in/spoken-tutorials}}}
\end{itemize}

\vskip5ex  

\section{About us}
\subsection{Website:}
\begin{center}
{\color{blue} {\fontfamily{lmss}\selectfont http://python.fossee.in}}
\end{center}  

\section{Contact us}

\subsection{General help \& Queries:}
\begin{center}
{\fontfamily{lmss}\selectfont info@fossee.in} \\
{\fontfamily{lmss}\selectfont python@fossee.in}
\end{center}



%==============================================================================
%==End of content==============================================================
%==============================================================================
%--References------------------------------------------------------------------
%--End of references-----------------------------------------------------------

\end{multicols}

%==============================================================================
\end{frame}
\end{document}
