% Created 2013-11-26 Tue 15:20
\documentclass[11pt]{article}
\usepackage[utf8]{inputenc}
\usepackage[T1]{fontenc}
\usepackage{fixltx2e}
\usepackage{graphicx}
\usepackage{longtable}
\usepackage{float}
\usepackage{wrapfig}
\usepackage{soul}
\usepackage{textcomp}
\usepackage{marvosym}
\usepackage{wasysym}
\usepackage{latexsym}
\usepackage{amssymb}
\usepackage{hyperref}
\tolerance=1000
\usepackage{minted}
\providecommand{\alert}[1]{\textbf{#1}}

\title{Using GIT}
\author{sachin}
\date{2013-11-26 Tue}
\hypersetup{
  pdfkeywords={},
  pdfsubject={GIT cheat-sheet},
  pdfcreator={Emacs Org-mode version 7.9.3f}}

\begin{document}

\maketitle

\setcounter{tocdepth}{3}
\tableofcontents
\vspace*{1cm}

\section{Install}
\label{sec-1}


\begin{minted}[]{sh}
sudo apt-get install/update git
\end{minted}
\section{Usage}
\label{sec-2}
\subsection{Elementary usage}
\label{sec-2-1}

    
\begin{itemize}
\item Create a directory

\begin{minted}[]{sh}
mkdir project1
\end{minted}
\item Initialize it as a git project 

\begin{minted}[]{sh}
git init
\end{minted}
\item Check the status of project

\begin{minted}[]{sh}
git status
\end{minted}
\item Add file(s) to \texttt{git}

     Add single file.

\begin{minted}[]{sh}
git add <FILENAME>
\end{minted}
     
     Add all files

\begin{minted}[]{sh}
git add .
\end{minted}

     Add all file in present directory

\begin{minted}[]{sh}
git add *
\end{minted}
\item Make \texttt{commit}
     
     With one line message

\begin{minted}[]{sh}
git commit -m "MESSAGE"
\end{minted}

     Open default text editor to write descriptive message.

\begin{minted}[]{sh}
git commit
\end{minted}

     Add and commit at the same time

\begin{minted}[]{sh}
git commit -am "MESSAGE"
\end{minted}
\item View logs(commits)

\begin{minted}[]{sh}
git log
\end{minted}
\item View \texttt{diff} between unstage file(s)

     Entire \texttt{diff}

\begin{minted}[]{sh}
git diff
\end{minted}

     File specific \texttt{diff}

\begin{minted}[]{sh}
git diff <FILENAME>
\end{minted}
\item Show commit hash specific state of a file.

\begin{minted}[]{sh}
git show <COMMIT HASH>:<FILENAME>
\end{minted}
\item Reset to previous commit

\begin{minted}[]{sh}
git reset --hard <COMMIT HASH>
\end{minted}
\end{itemize}
\subsection{Intermediate usage}
\label{sec-2-2}

\begin{itemize}
\item Branching

     Create branch

\begin{minted}[]{sh}
git branch <BRANCH NAME>
\end{minted}

     Delete branch

\begin{minted}[]{sh}
git branch -d <BRANCH NAME>
\end{minted}

     Change branch

\begin{minted}[]{sh}
git checkout <BRANCH NAME>
\end{minted}

     Merge branch

\begin{minted}[]{sh}
git merge <BRANCH NAME>
\end{minted}
\end{itemize}
\subsection{Remotes}
\label{sec-2-3}

\begin{itemize}
\item Manage git remotes

      View remote host

\begin{minted}[]{sh}
git remote -v
\end{minted}

      Add remote

\begin{minted}[]{sh}
git remote add <REMOTE NAME> <REMOTE URL>
\end{minted}
      Example:

\begin{minted}[]{sh}
git remote add origin https://github.com/iitbaakash/project1.git
\end{minted}
    
      Remove remote

\begin{minted}[]{sh}
git remote remove <REMOTE NAME>
\end{minted}
      Example

\begin{minted}[]{sh}
git remote remove origin
\end{minted}
\end{itemize}
\subsection{Push/Pull}
\label{sec-2-4}

\begin{itemize}
\item Manage push/pull

      Push source for the first time

\begin{minted}[]{sh}
git push -u <REMOTE NAME> <BRANCH NAME>
\end{minted}
    
      and then

\begin{minted}[]{sh}
git push
\end{minted}

      Pull source from remote(assuming that the remote is already been added)

\begin{minted}[]{sh}
git pull
\end{minted}
\end{itemize}
    

\end{document}
