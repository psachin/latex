% Created 2013-10-19 Sat 16:38
\documentclass[11pt]{article}
\usepackage[utf8]{inputenc}
\usepackage[T1]{fontenc}
\usepackage{fixltx2e}
\usepackage{graphicx}
\usepackage{longtable}
\usepackage{float}
\usepackage{wrapfig}
\usepackage{soul}
\usepackage{textcomp}
\usepackage{marvosym}
\usepackage{wasysym}
\usepackage{latexsym}
\usepackage{amssymb}
\usepackage{hyperref}
\tolerance=1000
\usepackage{minted}
\providecommand{\alert}[1]{\textbf{#1}}

\title{Clojure}
\author{sachin}
\date{2013-09-26 Thu}
\hypersetup{
  pdfkeywords={clojure, emacs, repl},
  pdfsubject={introduction to Clojure},
  pdfcreator={Emacs Org-mode version 7.9.3f}}

\begin{document}

\maketitle

\setcounter{tocdepth}{3}
\tableofcontents
\vspace*{1cm}

\section{lein}
\label{sec-1}

\begin{itemize}
\item Download \textbf{lein} binary file from \href{https://raw.github.com/technomancy/leiningen/stable/bin/lein}{github} and copy it to your
     \~{}/bin/ directory.
\item Add \~{}/bin PATH in your \~{}/.bashrc
\item Make the binary executable
\end{itemize}

\begin{minted}[]{sh}
chmod a+x ~/bin/lein
\end{minted}

\begin{itemize}
\item Now run the self installer

\begin{minted}[]{sh}
lein
\end{minted}
\end{itemize}
\subsection{Lein configuration}
\label{sec-1-1}

\begin{itemize}
\item lein version 2 uses the idea of \emph{profiles}. The location of file
      is \~{}/.lein/profiles.clj
\item sample profiles.clj file

\begin{minted}[]{clojure}
1:  {:user {:plugins [[lein-difftest "1.3.8"]
2:                    [lein-marginalia "0.7.1"]
3:                    [lein-pprint "1.1.1"]
4:                    [lein-swank "1.4.5"]
5:                    [lein-exec "0.3.0"]
6:                    [lein-ring "0.8.7"]]}}
\end{minted}
\end{itemize}
\subsection{Upgrade}
\label{sec-1-2}

\begin{itemize}
\item Upgrade lein to latest stable version using

\begin{minted}[]{sh}
lein upgrade
\end{minted}
\end{itemize}
\section{project/app}
\label{sec-2}
\subsection{create}
\label{sec-2-1}

\begin{itemize}
\item create a project using

\begin{minted}[]{sh}
lein new my-project
\end{minted}
\item create an app using
      

\begin{minted}[]{sh}
lein new app my-stuff
\end{minted}
\item create a noir project(\texttt{noir} is a plugin web framework)

\begin{minted}[]{sh}
lein new noir my-web-project
\end{minted}
\end{itemize}
\subsection{REPL}
\label{sec-2-2}

\begin{itemize}
\item Go to your project directory and run \texttt{lein}

\begin{minted}[]{sh}
cd my-stuff
lein repl
\end{minted}
\item inside \texttt{repl} prompt, run \texttt{(-main)}

\begin{minted}[]{clojure}
my-stuff.core=> (-main)
\end{minted}
\item If you have done some change in a code, reload it

\begin{minted}[]{clojure}
my-stuff.core=> (require 'my-stuff.core :reload)
\end{minted}
\item run a program using \texttt{lein}

\begin{minted}[]{sh}
lein run
\end{minted}
\item if you are happy with your app, package it into an executable
      \texttt{jar}

\begin{minted}[]{sh}
lein uberjar
\end{minted}
\item and run \texttt{jar} file using

\begin{minted}[]{sh}
java -jar target/my-stuff-0.1.0-standalone.jar
\end{minted}
\end{itemize}
\section{library}
\label{sec-3}
\subsection{create}
\label{sec-3-1}


\begin{minted}[]{sh}
lein new default my-lib
\end{minted}
\subsection{REPL}
\label{sec-3-2}


\begin{minted}[]{sh}
$ lein repl
...
\end{minted}
    

\begin{minted}[]{clojure}
user=> (require 'my-lib.core)
nil
user=> (ns my-lib.core)
nil
my-lib.core=> (my-func 3)
9
\end{minted}
\section{dependencies}
\label{sec-4}

\begin{itemize}
\item add project description to \~{}/.lein/profiles.clj or app/project.clj


\begin{minted}[]{clojure}
1:  {:user {:plugins [[lein-difftest "1.3.8"]
2:                    [lein-marginalia "0.7.1"]
3:                    [lein-pprint "1.1.1"]
4:                    [lein-swank "1.4.5"]
5:                    [lein-exec "0.3.0"]
6:                    [lein-ring "0.8.7"]]}}
\end{minted}
\item or you can have it specific to the project

\begin{minted}[]{clojure}
 1:  (defproject perfect-clojure "0.1.0-SNAPSHOT"
 2:    :description "A simple clojure app to test my environment"
 3:    :url "http://clojuremadesimple.co.uk"
 4:    :license {:name "Eclipse Public License"
 5:              :url "http://google.co.uk"}
 6:    :dependencies [[org.clojure/clojure "1.3.0"]]
 7:    :dev-dependencies [[midje "1.4.0"]
 8:                       [autodoc "0.9.0"]
 9:  
10:    :plugins [[lein-swank "1.4.4"]]]
11:    )
\end{minted}
\end{itemize}

   
\begin{itemize}
\item set or unset proxy and run

\begin{minted}[]{sh}
lein deps
\end{minted}
\end{itemize}
\section{compiling}
\label{sec-5}

\begin{itemize}
\item Build an executable \texttt{jar}

\begin{minted}[]{sh}
lein deps
lein uberjar
\end{minted}
\end{itemize}
    
\section{running}
\label{sec-6}


\begin{minted}[]{sh}
java -jar target/my-stuff-0.1.0-SNAPSHOT-standalone.jar
\end{minted}
\section{connecting to REPL server}
\label{sec-7}

\begin{itemize}
\item A new REPL server is started at certain PORT when we invoke

\begin{minted}[]{sh}
lein repl
\end{minted}
\item We can connect to existing server using

\begin{minted}[]{sh}
lein repl :connect nrepl://localhost:PORT
\end{minted}
\end{itemize}
\section{generating docs from libraries}
\label{sec-8}

\begin{itemize}
\item install \texttt{marginalia}

\begin{minted}[]{sh}
cd ~/.lein
touch profiles.clj
\end{minted}
\item add following line to \texttt{profiles.clj} (\texttt{marginalia} version may be
     different)

\begin{minted}[]{clojure}
{:user {:plugins [[lein-marginalia "0.7.1"]]}}
\end{minted}
\item then, in your project 

\begin{minted}[]{sh}
cd /path/to/project/
\end{minted}
\item generate docs

\begin{minted}[]{sh}
lein marg
\end{minted}
\item browse docs in path using web-browser

\begin{minted}[]{sh}
file://path/to/my-proj/docs/uberdoc.html
\end{minted}
\end{itemize}
\section{notes}
\label{sec-9}
\subsection{Difference between :use and :require}
\label{sec-9-1}


\begin{minted}[]{clojure}
(ns whatever
  (:use [some.namespace :only [vars you want]]))
\end{minted}


\begin{minted}[]{clojure}
(ns whatever
  (:require [some.namespace :as sn]))
\end{minted}
\section{references}
\label{sec-10}

\begin{itemize}
\item \href{http://www.unexpected-vortices.com/clojure/brief-beginners-guide/}{http://www.unexpected-vortices.com/clojure/brief-beginners-guide/}
\item \href{http://clojuremadesimple.co.uk/}{http://clojuremadesimple.co.uk/}
\end{itemize}

\end{document}
